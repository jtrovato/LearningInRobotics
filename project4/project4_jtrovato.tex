% ESE650 Learning in Robotics - Project 4

\documentclass{article}
\usepackage{graphicx}
\usepackage{color}
\usepackage{listings}
\usepackage{fullpage}
\usepackage{amsmath}
\usepackage{placeins}

\definecolor{lightgray}{gray}{0.5}
\setlength{\parindent}{0pt}
\setcounter{secnumdepth}{0} %turn off section numbering
\begin{document}

\title{ESE650 Project 4: Simultaneous Localization and Mapping}
\author{Joe Trovato}
\date{\today}
\maketitle
\setlength{\parindent}{10ex}

\section{Introduction}
Simultaneous and Localization and Mapping (SLAM) is a core problem in robotics. The ability to sense the surrounding environment and determine where the robot is within that environment is essential to a variety of tasks. In this project, I implemented a basic SLAM algorithm using a particle filter to track location and an probabilistic occupancy grid to keep track of the map. Once SLAM was working (to an acceptable level of accuracy), I used the RGB data taken from the camera to map color to the floor displayed in my occupancy grid. 


\section{Algorithm}
\subsection{Data Preprocessing}
The dataset given contained a variety of information, most of which  I did not incorporate into the project. My implementation uses the lidar readings, the head and neck joint angles, and their respective time stamps. For the final part of the project I used the RGB data contained in the kinect data file to map color to the floor. 
 
\subsection{Mapping}
After experimenting with naive algorithm for a binary occupancy grid, I implemented a logistic occupancy grid which tracked a probability of being occupied for each grid square. 

\subsection{Localization}
\subsubsection{Particle Filter}

\section{Results}


\section{Demo Instructions}
\begin{enumerate}
	\item Specify test files for lidar and joints by editing files names at top of SLAM.m
	\item change dataset variable at top of SLAM.m to correct number
	\item run SLAM.m, map should display

\end{enumerate}


\end{document}